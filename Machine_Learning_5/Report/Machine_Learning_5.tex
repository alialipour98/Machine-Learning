\documentclass[DIV=calc, paper=a4, fontsize=11pt, twocolumn]{scrartcl}	 % A4 paper and 11pt font size

\usepackage{multirow}
\usepackage{graphicx}
\usepackage{lipsum} % Used for inserting dummy 'Lorem ipsum' text into the template
\usepackage[english]{babel} % English language/hyphenation
\usepackage[protrusion=true,expansion=true]{microtype} % Better typography
\usepackage{amsmath,amsfonts,amsthm} % Math packages
\usepackage[svgnames]{xcolor} % Enabling colors by their 'svgnames'
\usepackage[hang, small,labelfont=bf,up,textfont=it,up]{caption} % Custom captions under/above floats in tables or figures
\usepackage{booktabs} % Horizontal rules in tables
\usepackage{fix-cm}	 % Custom font sizes - used for the initial letter in the document

\usepackage{sectsty} % Enables custom section titles
\allsectionsfont{\usefont{OT1}{phv}{b}{n}} % Change the font of all section commands

\usepackage{fancyhdr} % Needed to define custom headers/footers
\pagestyle{fancy} % Enables the custom headers/footers
\usepackage{lastpage} % Used to determine the number of pages in the document (for "Page X of Total")

% Headers - all currently empty
\lhead{}
\chead{}
\rhead{}

% Footers
\lfoot{}
\cfoot{}
\rfoot{\footnotesize Page \thepage\ of \pageref{LastPage}} % "Page 1 of 2"

\renewcommand{\headrulewidth}{0.0pt} % No header rule
\renewcommand{\footrulewidth}{0.4pt} % Thin footer rule

\usepackage{lettrine} % Package to accentuate the first letter of the text
\newcommand{\initial}[1]{ % Defines the command and style for the first letter
\lettrine[lines=3,lhang=0.3,nindent=0em]{
\color{DarkGoldenrod}
{\textsf{#1}}}{}}

%----------------------------------------------------------------------------------------
%	TITLE SECTION
%----------------------------------------------------------------------------------------

\usepackage{titling} % Allows custom title configuration

\newcommand{\HorRule}{\color{DarkGoldenrod} \rule{\linewidth}{1pt}} % Defines the gold horizontal rule around the title

\pretitle{\vspace{-30pt} \begin{flushleft} \HorRule \fontsize{20}{20} \usefont{OT1}{phv}{b}{n} \color{DarkRed} \selectfont} % Horizontal rule before the title

\title{Machine Learning (Problem set 5)} % Your article title

\posttitle{\par\end{flushleft}\vskip 0.5em} % Whitespace under the title

\preauthor{\begin{flushleft}\large \lineskip 0.5em \usefont{OT1}{phv}{b}{sl} \color{DarkRed}} % Author font configuration

\author{Ali Alipour, } % Your name

\postauthor{\footnotesize \usefont{OT1}{phv}{m}{sl} \color{Black} % Configuration for the institution name
University of Tehran % Your institution

\par\end{flushleft}\HorRule} % Horizontal rule after the title

\date{} % Add a date here if you would like one to appear underneath the title block

%----------------------------------------------------------------------------------------

\begin{document}

\maketitle % Print the title

\thispagestyle{fancy} % Enabling the custom headers/footers for the first page 

%----------------------------------------------------------------------------------------
%	ABSTRACT
%----------------------------------------------------------------------------------------

% The first character should be within \initial{}
\initial{T}\textbf{his report discusses methods for selecting optimal features and performing dimensionality 
                   reduction using Principal Component Analysis (PCA) and Linear Discriminant Analysis (LDA). 
                   The report also includes model performance evaluation through cross-validation with different n
                   umbers of components.}
%----------------------------------------------------------------------------------------
%	ARTICLE CONTENTS
%----------------------------------------------------------------------------------------

\section{\small{Feature Selection and Dimensionality Reduction}}

   \subsection{Principal Component Analysis (PCA)}
      PCA helps identify the optimal number of components by balancing the retention of meaningful information and 
      reducing dimensionality. The method finds an "elbow point" where the eigenvalues decrease significantly, 
      indicating that most of the variance is captured by the components before this point.

   \subsection{\small{Cumulative Variance}}
      The cumulative variance plot helps visualize how much total variance is explained as more components are added. 
      This ensures that we keep enough components to capture the majority of the data's variance.

   \subsection{\small{Model Evaluation Using Cross-Validation}}
      Cross-validation is used to evaluate the performance of the model with different numbers of components. 
      It helps determine the number of components that yield the best accuracy.

\section{\small{Linear Discriminant Analysis (LDA)}}
   LDA focuses on maximizing class separability by applying a linear transformation. It selects the components 
   that contribute the most to between-class variance while minimizing within-class variance.

   \subsection{\small{Feature Selection in LDA}}
      LDA automatically selects the most important features that maximize class separability. 
      This selection process enhances classification accuracy and reduces the time required for convergence.

\section{\small{Evaluation of Gaussian Mixture Models (GMM)}}
   Gaussian Mixture Models (GMM) are used to classify the dataset. Cross-validation helps determine the 
   optimal number of components for the GMM, ensuring that the model minimizes classification errors.

\section{\small{Question 1}}
   The evaluation of different techniques for dimensionality reduction is crucial in ensuring that meaningful 
   information is preserved while minimizing unnecessary data. Methods like PCA and LDA offer distinct advantages 
   depending on the dataset and the goals of the analysis.

\section{\small{Question 2}}
   Cross-validation serves as a powerful tool for evaluating model performance across different configurations, 
   including the number of components or features selected. By comparing results obtained from various settings, 
   we can choose the configuration that offers the best trade-off between complexity and accuracy.

%----------------------------------------------------------------------------------------
%	REFERENCE LIST
%----------------------------------------------------------------------------------------

\begin{thebibliography}{99} % Bibliography - this is intentionally simple in this template

  \bibitem[Alipour Fraydani, 2024]{AlipourFraydani:2024}
  Alipour Fraydani, A. (2024).
  \newblock Homework on Machine Learning problem set 5, University of Tehran.
  \newblock {\em Unpublished Manuscript}, Department of Electrical Engineering, University of Tehran.
  
\end{thebibliography}

%----------------------------------------------------------------------------------------

\end{document}