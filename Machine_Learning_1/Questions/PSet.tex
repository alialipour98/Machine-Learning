\documentclass[12pt]{article}

\usepackage{fullpage}
\usepackage{epic}
\usepackage{eepic}
\usepackage{graphicx}

\newcommand{\proof}[1]{
{\noindent {\it Proof.} {#1} \rule{2mm}{2mm} \vskip \belowdisplayskip}
}


\newtheorem{lemma}{Lemma}[section]
\newtheorem{theorem}[lemma]{Theorem}
\newtheorem{claim}[lemma]{Claim}
\newtheorem{definition}[lemma]{Definition}
\newtheorem{corollary}[lemma]{Corollary}

\setlength{\oddsidemargin}{0in}
\setlength{\topmargin}{0in}
\setlength{\textwidth}{6.5in}
\setlength{\textheight}{8.5in}

\begin{document}

\setlength{\fboxrule}{.5mm}\setlength{\fboxsep}{1.2mm}
\newlength{\boxlength}\setlength{\boxlength}{\textwidth}
\addtolength{\boxlength}{-4mm}
\begin{center}\framebox{\parbox{\boxlength}{\bf
Machine Learning \hfill 
}}\end{center}
\vspace{5mm}

\section*{Question 1}
The objective of this question is to familiarize you with the Naïve Bayes classifier and its implementation.
\begin{itemize}
    \item \textbf{Part A:} Explain the Naïve Bayes classifier and its structural differences compared to 
                           the optimal Bayes classifier. Explain why it makes sense to use Naïve Bayes instead 
                           of the optimal Bayes classifier, particularly in cases where assuming independence 
                           between features might not be strictly true.
    
    \item \textbf{Part B:} Implement a Naïve Bayes classifier from scratch for a dataset with three classes. 
                           You need to build the classifier for each class individually, and evaluate accuracy, 
                           precision, recall, and generate a confusion matrix.

    \item \textbf{Part C:} Implement the Naïve Bayes classifier using SKLearn and compare the results of part (b) 
                           with the results obtained using SKLearn.
\end{itemize}

\section*{Question 2}
Design a binary classifier to distinguish between sea and forest images using the 'image' 
dataset. Report accuracy, precision, recall, and confusion matrix.

\begin{itemize}
    \item \textbf{Hint:} You don't need to use any famous feature extraction algorithms for classification. 
                         You can use color features to separate the two classes (for instance, the sea images are 
                         more likely to have blue tones, while the forest images are more likely to have green tones).
    
\end{itemize}

% \begin{theorem}
% This is a theorem statement.
% \label{thm:sample-statement}
% \end{theorem}

% \proof{
% This is a proof.
% }



\end{document}
